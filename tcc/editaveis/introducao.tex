\chapter*[Introdução]{Introdução}
\section{Preâmbulo}
Parkinson
sEMG
aprendizado de máquinas
sVM
\section{Justificativa}
Atualmente o diagnóstico da Doença de Parkinson é basicamente clinico onde o Neurologista identifica alguns sintomas associados a doença \cite{gago2014manual}, assim sendo, este sistema visa auxiliar e facilitar o processo de diagnóstico utilizando um único exame não invasivo.

\section{Objetivos}
\subsection{Objetivo Geral}
O Objetivo deste Trabalho é o desenvolvimento de um \textit{Software} utilizando algorítimos de aprendizado de máquinas capazes de auxiliar no diagnóstico da Doença de Parkinson, com dados obtidos por um dispositivo de coleta de sinais sEMG.

\subsection{Objetivos Específicos}
\begin{itemize}
    \item Analisar pesquisas semelhantes.
    \item Entender o sinal sEMG relacionado a DP.
    \item Definir \textit{features} para análise no \textit{Software}.
    \item Tratar os dados para otimizar o processo de AM.
    \item Escolher algorítimos para resolver o problema de AM.
    \item Treinar os algorítimos para a predição.
    \item Validar a o modelo.
    \item Definir arquitetura do \textit{Software}.
    \item Desenvolver um sistema integrando o algorítimo treinado anteriormente.
    \item Testar o \textit{software} desenvolvido.
\end{itemize}
\section{Trabalhos correlatos}
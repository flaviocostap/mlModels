\chapter[Solução de software]{Solução de software}
Esse projeto utilizará a metódologia ágil de desenvolvimento de software com uma adaptação do \textit{Scrum} para projetos pequenos, para a organização e desenvolvimento do trabalho como um todo.

\section{Ágil}
\subsection{Scrum}
O \textit{Scrum} é uma metodologia ágil para o gerenciamento de projeto de software, consiste na estruturação de um projeto em pequenas fases incrementais, com prazo definido e escopo variável de acordo as demandas e necessidades do cliente. As iterações são chamados de \textit{Sprints} e normalmente vão de uma semana a um mês, dependendo da organização e necessidades do projeto. As funcionalidades do software a serem desenvolvidas, são colocadas em uma lista denominada \textit{Product Backlog}. Em cada \textit{Sprint} as funcionalidade são priorizas e selecionas, e após são alocadas na lista de \textit{Sprint Backlog}, as quais seram desenvolvidas na \textit{Sprint} \cite{sutherland2016scrum}.

Durante as \textit{Sprints}, devem ocorrer reuniões diárias, as quais necessitam ser objetivas e rápidas, com tempo fechado, não sendo ideal ultrapassar 15 minutos. Nessa reunião, busca-se alinhar o desenvolvimento do projeto entre toda a equipe, onde todos os integrantes devem responder as seguintes questões \cite{sutherland2016scrum}.:
\begin{itemize}
    \item \textit{O que eu fiz ontem?}
    \item \textit{O que eu farei hoje?}
\end{itemize}

Ao final de cada \textit{Sprint}, devem ocorrer uma reunião de retrospectiva, onde a equipe planejará a nova \textit{Sprint} \cite{sutherland2016scrum}.

\section{Ferramentas}
Como descrito na seção ~\ref{sec:MAFerramentas} na página \pageref{sec:MAFerramentas}, uma das principais linguagens de programação em MA é o Python, com a biblioteca \textit{scikit-learn}, com este cenário em vista, optou-se por desenvolver o resto da aplicação em Python com, para assim ter uma boa integração com o modelo desenvolvido.

\subsection{scikit-learn}
O \textit{scikit-learn} será o \textit{framework} utilizado para o desenvolvimento dos modelos de AM;
\subsection{Django-ResT}

A Django Rest Framework é uma blibioteca Django que viabiliza de forma simples a criação de APIs REST em projetos django usando de meios já conhecidos dos desenvolvedores para proporcionar produtividade na criação das APIS \cite{christie2011django}.

\subsection{React}

A React é uma biblioteca JavaScript declarativa, eficiente e flexível para a criação de interfaces de usuário (UI). Ou seja, ela é simplesmente uma coleção de funcionalidades relacionadas que podem ser chamadas pelo desenvolvedor para resolver problemas específicos, sendo esses a criação de interfaces de usuário reaproveitáveis \cite{reactjs}.

\chapter[TrabalhosRelacionados]{Trabalhos Relacionados}

Existem diversos trabalhos associados a identificação da doença de Parkinson e aprendizado de máquinas, como em \cite{camara2015resting} que utilizou uma rede neural artificial para identificar a doença através do Tremor em repouso, \citeonline{ai2011classification} utilizou sEMG com SVM para identificar a doença de Parkinson e o Tremor essencial, ele utilizou uma a decomposição em modo empírico (do inglês,\textit{empirical mode decomposition} \textbf{EMD} ) para decompor o sinal em funções do modo intrínseco (\textbf{IMF}) e após a decomposição em valores singulares ou \textit{singular value decomposition} (\textbf{SVD}) para extrair as \textit{features} das IMF geradas, e em seguida inseridas no SVM, \citeonline{ai2011classification} também comparou o desempenho substituindo a EMD pela transformada discreta de \textit{wavelets} (\textbf{DWT})  e verificou através de validação cruzada que o método EMD-SVD foi superior ao método DWT-SVD.

SVM também foi utilizada por \cite{kugler2013automated} que combinou sEMG com sinais de acelerômetro para identificar a doença de Parkinson e o Tremor essencial, para validação foi utilizado a validação cruzada.

Outro trabalho interessante foi realizado por \cite{loconsole2018model}, que propôs uma técnica em 3 estágios para diferenciar portadores da DP, usando analise caligráfica com sEMG, e como classificador foi utilizado uma rede neural artificial alinhada com o SVM.

Um comparativo sobre as estratégias de redução de dimensionalidade das \textit{features} em EMG foi realizado por \cite{liu2014feature} obtendo uma média de 95\% de classificação em 12 \textit{features} selecionadas. 
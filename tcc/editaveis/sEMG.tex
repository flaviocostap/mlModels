\chapter{Eletromiografia de Superfície - sEMG}
\section{Definição}
A eletromiográfia de Superfície é o estudo relacionado as transformações elétricas referentes as contrações musculares. É um exame indolor e não invasivo permitindo assim a execução com mobilidade dos movimentos musculares solicitados, podendo ser execultada repetidas vezes sem causar um grande desconforto ao paciente, sendo rápida, barata, livre de radiação, não invasivo e de fácil compreensão, podendo ser utilizado na análise de um grupo ou um feixe muscular específico \cite{de2010eletromiografia}.

A sEMG caracteriza-se pela utilização de um dispositivo sobre a pele do paciente, o qual implica a detecção dos potências elétricos relativos as fibras musculares, ou seja é possivel detectar quando um musculo é ativado e qual o movimento foi execultado, e ainda relacionar o associação dos diferentes musculos envolvidos \cite{botelho2010avaliaccao}.

O sinal EMG é registrado normalmente por eletrodos de superfície, mas pode também ser utilizado eletrodos de agulha \cite{soderberg1984electromyography}.

\section{Características do sinal}

Um sEMG comum varia entre 0.1 a 0.5 millivolt, podendo conter frequência de até 10 kHz, a amplitude do sinal depende de vários fatores como o tipo de eletrodo utilizado, nivel do esforço muscular, colocação dos eletrodos.

\section{Usos}
A sEMG é largamente utilizada por todos pelas diversas áreas cientificas que estudam o movimento humano, como Médicos, Fisioterapeutas, Fonoaudiólogos e profissionais em Educação Física\cite{nascimento2012surface}.

\begin{resumo}[Abstract]

 \begin{otherlanguage*}{english}
  Parkinson's disease is a neurological disease that affects a large part of the world population, being mostly elderly people, although the effective reasons that cause the disease have not been discovered, its diagnosis is possible thanks to the evident signs caused by the presence of the disease, despite the means to validate this diagnosis are already given as sufficient there is still a need to obtain more information about the state of the patient with this disease for possible studies on it. To guarantee access to this information and to assist in the diagnosis of Parkinson's Disease, it is possible to use the concepts of Machine Learning that have been widely requested in several areas of knowledge, including health, for the solution of problems in different contexts. This course completion work aims to implement a Machine Learning model that assists in the diagnosis of Parkinson's Disease using samples collected by sEMGs devices in order to recognize patterns and obtain new information about the disease in question, using a proposed methodology of construction of machine learning system.

   \vspace{\onelineskip}
 
   \noindent 
   \textbf{Key-words}: Parkinson. Machine Learning. sEMG. Health.
 \end{otherlanguage*}
\end{resumo}

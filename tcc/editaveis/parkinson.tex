\chapter{Parkinson}
\section{Definição}
A Doença de Parkinson (DP) é uma doença crônica e
degenerativa do sistema nevorso central, caracterizada principalmente por distubios motores \cite{souzametodos}, como a bradicinesia 
(redução da movimentação, o paciente consegue se movimentar porém lentamente), tremor em repouso, rigidez corporal, além de dificuldade na 
fala o qual atinge aproximadamente 90\% dos pacientes\cite{da2016aspectos}.

\section{Diagnóstico e Sintomas}
A doença de Parkinson afeta aproximadamente entre 1\% a 2\% da população mundial acima dos 65 anos, sendo que no 
Brasil atinge entorno de 3\% da população nesta faiza etaria.  
\cite{magalhaes2009descobrindo}, tendo a idade como o único fator de risco conhecido, a DP é rara antes dos 40 anos, 
aumenta após os 50 e é máxima após os 70 anos
 \cite{peixinho2006alteraccoes}, porém a incidência da doença não esta restrita somente a pessoas idosas,
  uma vez que 20\% dos casos são de pessoas com menos de 50 anos. \cite{gago2014manual}

Com relação a incidência por sexo não há concenco estabelecido, mas alguns estudos relacionam ser um pouco maior a ocorrência 
no sexo masculino \cite{peixinho2006alteraccoes}.

O diagnóstico da DP atualmente é clinico, onde identifica-se a bradicinésia e pelo menos um de três outros sintomas,
 tremor em repouso, rigidez ou instabilidade corporal \cite{peixinho2006alteraccoes}.
\section{DP e sEMG}
\section{DP e machine learning}


A idade é o único factor de risco confirmado:
a DP é rara antes dos 40 anos, aumenta
após os 50 e é máxima a partir dos 70 anos,
embora esteja descrito um ligeiro decréscimo
no grupo etário superior aos 80 anos4,5.
